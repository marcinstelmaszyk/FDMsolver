\chapter{Wprowadzenie}
\section{Opis projektu}

\indent\indent Odległość od brzegu, $d$, pozostaje nadal kluczowym parametrem przy modelowaniu turbulencji oraz generowaniu siatek obliczeniowych\footcite{Tucker}.

\indent Celem niniejszego projektu jest opracowanie programu  umożliwiającego wyznaczanie minimalnej odległości między danym węzłem siatki, a brzegiem profilu lotniczego.
 
Najprostszym sposobem rozwiązania powyższego zagadnienia jest  bezpośrednie obliczenie odległości $d$ danego węzła $P$ od kolejnych węzłów $S$ należących do brzegu profilu korzystając z miary euklidesowej:

\begin{equation}
d(P,S)= \sqrt{(x_p-x_s)^2 + (y_p-y_s)^2}
\end{equation}

\noindent a następnie wybranie najmniejszej wartości z utworzonego zbioru odległości $\{d_0, d_1, ..., d_n\}$. Metoda ta, znana w terminologii algorytmicznej jako \textit{metoda siłowa} (ang. \textit{Brute Force}), jest łatwa do zaimplementowania, a liczba operacji potrzebnych do jej wykonania dla siatek stałych w czasie jest mała w stosunku do kosztu całego rozwiązania. Niemniej jednak, szybkie znalezienie odległości od brzegu ma kluczowe znaczenie dla problemów rozwiązywanych przy użyciu siatek deformujących się i adaptacyjnych, dla których wyznaczenie minimalnej odległości od brzegu jest przeprowadzane wielokrotnie\footcite{Ibid}. 

Innym podejściem jest wykorzystanie równań różniczkowych cząstkowych. Do rozwiązania problemu Sethain\footcite{Sethain} zaproponował wykorzystanie równania Eikonału następującej postaci

\begin{equation}
\label{eq:eikonal}
\left|\nabla \phi\right|=1+\lambda\nabla^2\phi
\end{equation}

\newpage
\indent Niniejszy projekt skupi się na implementacji uproszczonej  jego wersji, a mianowicie równaniu Poissona na płaszczyźnie $(x,y)$

\begin{equation}
\nabla^2\phi = - 1
\end{equation}

\indent Pomocniczo została utworzona siatka obliczeniowa wokół profilu, o strukturze dopasowanej do zadanego brzegu obszaru. Tak zdefiniowana przestrzeń fizyczna została przetransformowana do przestrzeni obliczeniowej, a następnie zdyskretyzowana metodą różnic skończonych. Powstały układ równań algebraicznych rozwiązano metodą iteracyjną. Otrzymane wyniki skonfrontowano z rozwiązaniem siłowym.

\indent Program został zaimplementowany w języku C++ przy użyciu darmowego środowiska programistycznego Microsoft Visual Studio Express 2012\footnote{\url{www.microsoft.com/visualstudio/}}. Język C++ ma szerokie zastosowanie w programach obliczeniowych, więc jego wykorzystanie umożliwia integrację w istniejących lub przyszłych kodach obliczeniowych. Do wizualizacji wyników skorzystano z programu \textsf{Tecplot 360}\footnote{\url{www.tecplot.com}}.

\indent Niniejsza dokumentacja została stworzona w języku \LaTeX\quad przy użyciu edytora \textsf{TexMaker 3.5.2}\footnote{\url{http://www.xm1math.net/texmaker/}}.