\chapter{Opis interfejsu programu}

\indent\indent Niniejszy dodatek opisuje sposób korzystania ze stworzonego w ramach niniejszego projektu programu \textsf{wallDistanceSolver.exe}, oraz jego interfejsu, do własnego użytku.

\section{Dane wejściowe}

\indent\indent Danymi wejściowymi do programu są współrzędne profilu zapisane w dwóch kolumnach, kolejno dla kierunku x oraz y, z użyciem kropki jako separatora dziesiętnego (kod źródłowy \ref{list:dane_wejściowe}). Domyślna nazwa pliku to \texttt{\mbox{NACA\_0012.dat}}

\begin{lstlisting}[label=list:dane_wejściowe,caption=Format zapisu danych wejściowych]
	0.964244   0.006169
	0.947231   0.008434
	........   ........
	........   ........	
\end{lstlisting}

\noindent\newline Sugeruje się nie pozostawiać pustych wierszy, złamanych znakiem nowej linii (enterem) na końcu kolumn, aby uniknąć możliwego wczytania nic nie znaczących znaków. Należy również zwrócić uwagę, aby wartości w wierszach nie powtarzały się, bowiem spowoduje to utworzenie kilku węzłów w tym samym punkcie, co przełoży się na rozwiązanie błędnie wygenerowanego układu równań \ref{eq:poisson_macierz}.   

\section{Dane wyjściowe}

\indent\indent Danymi wyjściowymi z programu są współrzędne $(x,y)$ węzłów siatki oraz wartość odległości od brzegu $d$, zapisane w trzech kolumnach w kolejności jak powyżej, z użyciem kropki jako separatora dziesiętnego (kod źródłowy \ref{list:dane_wyjściowe}). Dodatkowo program umożliwia wstawienie nagłówka niezbędnego do wczytania danych do oprogramowania \texttt{Tecplot 360}\footnote{\url{ftp://ftp.tecplot.com/pub/doc/tecplot/360/dataformat.pdf}, s. 134}

\begin{lstlisting}[label=list:dane_wyjściowe,caption=Format zapisu danych wyjściowych]
\end{lstlisting}\begin{center}
\underline{Zwykłe formatowanie}
\end{center}
\begin{lstlisting}  
              1              0              0
       0.979641       0.004079              0
       0.964244       0.006169              0 
       ........       ........       ........
       0.850307       0.032659       0.012574
       0.831428       0.035881       0.013543
	   ........       ........	     ........	
\end{lstlisting}
\begin{center}
\underline{Format danych programu Tecplot 360}
\end{center}
\begin{lstlisting}  
	VARIABLES = "X", "Y", "U",
	ZONE I=124, J=41, DATAPACKING=POINT

              1              0              0
       0.979641       0.004079              0
       0.964244       0.006169              0 
       ........       ........       ........
       0.850307       0.032659       0.012574
       0.831428       0.035881       0.013543
	   ........       ........	     ........
\end{lstlisting}


\section{Przykładowy program}

\indent\indent W pierwszym kroku definiowane są zmienne pomocnicze, określające promień siatki \textsf{circleRadius}, liczbę rzędów \textsf{gridDensity} oraz położenie jej środka \textsf{S} (korzystając z  klasy pomocniczej \textsf{cPoint}).

Następnie następuje inicjalizacja obiektu solvera \textsf{CASE}, oraz wczytanie z pliku współrzędnych punktów profilu metodą \textsf{addProfile()}. Na podstawie tych danych tworzona jest siatka o wcześniej zdefiniowanych parametrach (metoda \textsf{generateGrid()}). Rozwiązanie zagadnienia odbywa się przy użyciu metody \textsf{solve(POISSON)}\footnote{Dla celów porównawczych można skorzystać z metody siłowej. W tym celu należy zmienić typ wyliczeniowy z \textsf{POISSON} na \textsf{BRUTE\_FORCE}.}

W ostatnim kroku otrzymane rozwiązanie jest zapisywane metodą \\ \mbox{\textsf{saveSolution()}}. Treść komunikatów programu przedstawiono w kodzie źródłowym (\ref{list:konsola}).

\begin{lstlisting}[label=list:program,caption=Przykładowy program rozwiązujący zagadnienie]
	#include <iostream>
	#include "cSolver.h"

	int main(int argc, char * argv[]){
		//Grid parametres
		cPoint	S				= cPoint(0.5, 0.0);	
		double	circleRadius	= 1;
		int		gridDensity		= 40;

		cSolver CASE;

		//Read profile from a file
		CASE.addProfile("../../data/NACA_0012.dat");

		//Generate grid based on profile
		CASE.generateGrid(S, circleRadius, gridDensity);

		//Solve problem
		std::cout << CASE.solve(POISSON) << std::endl;
	
		//Save solution to a file
		CASE.saveSolution(DISTANCE);	
	
		return 0;
	}
\end{lstlisting}


\begin{lstlisting}[label=list:konsola,caption=Widok konsoli programu]
	Generating grid
	Grid generated!
	Solving Poisson...
		#iterations: 81	estimated error: 1.83435e-06
	Solution solved!
	6.908
	Saving solution...
	Solution saved!
\end{lstlisting}
